
\documentclass[a4paper,12pt]{article}

\usepackage[a4paper, total={6in, 8in}, left=30mm]{geometry}
\usepackage{pdfpages}

\usepackage{cmap}
\usepackage[T2A]{fontenc}
\usepackage[utf8]{inputenc}
\usepackage[english,russian]{babel}
\usepackage{fancyhdr}
\usepackage{minted}
\usepackage{hyperref}
\usepackage{amsmath}
\usepackage{graphicx}

\hypersetup{
  pdfborderstyle={/S/U/W 1}
}

\graphicspath{{./images/}}

\pagestyle{fancy}
\fancyhf{}
\lhead{Антон Завьялов, ПИ-72}
\rhead{\textbf{Лабораторная №14}}
\cfoot{\thepage}

\makeatletter
\def\@seccntformat#1{%
  \expandafter\ifx\csname c@#1\endcsname\c@section\else
  \csname the#1\endcsname\quad
  \fi}
\makeatother

\begin{document} % Конец преамбулы, начало текста.
\includepdf[pages={1}]{title.pdf}

\section{\normalsize{Задание к лабораторной работе}}
\begin{flushleft}
    \begin{itemize}
        \item Составить программу решения нелинейной краевой задачи для дифференциального уравнения второго порядка методом стрельбы.
        \item Проанализировать полученный результат, исходя из физического смысла задачи, и при необходимости провести дополнительные вычисления.
    \end{itemize}
\end{flushleft}
\begin{flushleft}
    \textit{Вариант 7.}\linebreak
    Увеличение поверхности тела для выделения тепла в виде излучения имеет большое значение при проектировании сильноточных проводников, так как оно является
    единственным способом отвода излишков тепла. Чтобы масса теплоизлучателя была по возможности наименьшей, его делают в виде набора тонких кольцевых пластин
    с малым углом между боковыми гранями. Распределение температуры в такой пластине является решением уравнения

    \[\frac{d^2U}{dR^2} + \Bigg(\frac{1}{R + \rho} - \frac{\tg{\alpha}}{(1-R)\tg{\alpha} + \theta}\Bigg)\frac{dU}{dR} - \frac{\beta U^4}{(1-R)\tg{\alpha}+\theta} = 0\]

    с граничными условиями

    \[U(0) = 1, \frac{dU(1)}{dR} = 0,\]

    где 

    \[R = \frac{r - r_B}{r_T - r_B}, U = \frac{T}{T_B}, \theta = \frac{z_T}{r_T - r_B}, \rho = \frac{z_B}{r_T - r_B}.\]

    Здесь $\alpha$ $-$ угол между гранями пластины; $r, r_B, r_T$ и $z_T$ $-$ текущий радиус, радиус основания, радиус вершины и толщина пластины в вершине соответственно;
    $T$ и $T_B$ $-$ текущая температура и температура основания.\linebreak

    Найти относительную температуру $U$, если $\rho = 0.5$, $\theta = 0.05$, $\alpha = 6^{\circ}$, а $\beta = 0.4$.
\end{flushleft}

\section{\normalsize{Краткое описание метода, расчётные формулы}}
\begin{flushleft}
\end{flushleft}


\section{\normalsize{Текст программы с комментариями}}
Кроме расположенных ниже листингов, исходный код можно посмотреть по адресу: \url{https://github.com/andiogenes/differential-equations/tree/master/shooting-method}

\section{\normalsize{Тестовые примеры}}
\begin{flushleft}
\end{flushleft}

\end{document}
