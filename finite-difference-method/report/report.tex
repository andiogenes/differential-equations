
\documentclass[a4paper,12pt]{article}

\usepackage[a4paper, total={6in, 8in}, left=30mm]{geometry}
\usepackage{pdfpages}

\usepackage{cmap}
\usepackage[T2A]{fontenc}
\usepackage[utf8]{inputenc}
\usepackage[english,russian]{babel}
\usepackage{fancyhdr}
\usepackage{minted}
\usepackage{hyperref}
\usepackage{amsmath}

\hypersetup{
  pdfborderstyle={/S/U/W 1}
}

\pagestyle{fancy}
\fancyhf{}
\lhead{Антон Завьялов, ПИ-72}
\rhead{\textbf{Лабораторная №15, вариант 7}}
\cfoot{\thepage}

\makeatletter
\def\@seccntformat#1{
  \expandafter\ifx\csname c@#1\endcsname\c@section\else
  \csname the#1\endcsname\quad
  \fi}
\makeatother

\begin{document}
\includepdf[pages={1}]{title.pdf}

\section{\normalsize{Задание к лабораторной работе}}
\begin{flushleft}
  \textit{Постановка задачи}\linebreak
  Требуется найти приближенное значение решения следующей краевой задачи:

  \begin{equation}\label{eq:pos_1}
    \frac{d^2u(x)}{dx^2} + A(x)\frac{du(x)}{dx} + B(x)u(x) = C(x),~x \in [a,b],
  \end{equation}

  \begin{equation}\label{eq:pos_2}
    F_1u(a) + D_1\frac{du(a)}{dx} = E_1,
  \end{equation}

  \begin{equation}\label{eq:pos_2}
    F_2u(b) + D_2\frac{du(b)}{dx} = E_2.
  \end{equation}

  \textit{Задание к лабораторной работе}\linebreak
  \begin{itemize}
    \item По аналогии с описанным в параграфе 7.2 методом, вывести формулы для коэффициентов системы линейных алгебраических уравнений с трехдиагональной матрицей, которая получается для метода первого порядка аппроксимации.
    \item Составить программу для решения краевой задачи для линейного дифференциального уравнения второго порядка методом прогонки в случае первого и второго порядка аппроксимации.
    \item Провести вычисления, разбивая отрезок интегрирования на различное число частей, например, 25, 50, 100, 200.
    \item Проанализировать зависимость нормы разности между точным и приближенным решением от шага сетки для методов первого и второго порядка аппроксимации. Для этого составить таблицу или построить графики нормы разности для методов различного порядка аппроксимации. В качестве нормы можно взять C-норму. В этом случае норма функции равна максимальному значению модуля этой функции.
    \item Исследовать для Вашего варианта вопрос об устойчивости прогонки.
  \end{itemize}

  \textit{Для 7 варианта:}\linebreak
  \begin{equation}\label{eq:params}
    A(x) = \frac{4x}{2x+1}, (B(x) = \frac{-4}{2x+1}, C(x) = 0,
  \end{equation}
  \begin{equation*}
    a = 0, b = 1,~F_1 = 1, D_1 = 1, E_1 = 0,~F_2 = 2, D_2 = 1, E_2 = 3,
  \end{equation*}
  \begin{equation*}
    u(x) = x + e^{-2x}.
  \end{equation*}
\end{flushleft}

\section{\normalsize{Краткое описание метода, расчётные формулы}}
\begin{flushleft}
\end{flushleft}


\section{\normalsize{Текст программы с комментариями}}
Кроме расположенных ниже листингов, исходный код можно посмотреть по адресу: \url{https://github.com/andiogenes/differential-equations/tree/master/finite-difference-method}

\section{\normalsize{Тестовые примеры}}

\end{document}
